\documentclass[11pt]{article}
\usepackage{coling2014}
\usepackage{times}
\usepackage{url}
\usepackage{latexsym}
\usepackage{graphicx}
\usepackage{amsmath}
\usepackage{xcolor}
\usepackage{enumitem}
\usepackage{color}
\usepackage{textcomp} %for copyright and registered symbols
\usepackage{soul}
\usepackage{semantic}
\usepackage{setspace}%to squeeze space in enumitem and bibliography
\usepackage[compact]{titlesec}
%the following to squeeze the bibliography
\let\oldbibliography\thebibliography
\renewcommand{\thebibliography}[1]{%
  \oldbibliography{#1}%
  \setlength{\itemsep}{0pt}%
}


\usepackage{amssymb}%for \done and \crossed under itemize
\newcommand{\done}{\item[\checkmark]}
\newcommand{\crossed}{\item[$\times$]}

%\setlength\titlebox{5cm}

% You can expand the titlebox if you need extra space
% to show all the authors. Please do not make the titlebox
% smaller than 5cm (the original size); we will check this
% in the camera-ready version and ask you to change it back.

      
      
\title{Modelling the Semantics of Adjectives in the Ontology-Lexicon Interface}

%\author{John P. M\textsuperscript{c}Crae\\
%  Universit\"at Bielefeld \\
%  Bielefeld\\
%  Germany \\
%  {\tt \footnotesize jmccrae@cit-ec.uni-bielefeld.de} \\ \And
%  Francesca Quattri \\
%  Hong Kong Polytechnic University \\
%  Hong Kong \\
%  {\tt \footnotesize francesca.quattri@connect.polyu.hk} \\ \And
%  Christina Unger, Philipp Cimiano \\
%  Universit\"at Bielefeld \\
%  Bielefeld\\
%  Germany \\
%  {\tt \footnotesize \{cunger,cimiano\}@cit-ec.uni-bielefeld.de}}

\author{Joe Bloggs\\
  Affiliation / Address line 1 \\
  Affiliation / Address line 2 \\
  Affiliation / Address line 3 \\
  {\tt email@domain}}


\date{}

\begin{document}
\maketitle
\begin{abstract}
The modelling of the semantics of adjectives is notoriously challenging. We consider this problem in the context of the so called ontology-lexicon interface which attempts to capture the semantics of words by reference to an ontology in description logics or some other, typically first-order, logical formalism.
The use of first order logic (hence also description logics),
while effective for nouns and verbs, breaks down in the case of adjectives. 
We argue that this is primarily due to a lack of logical expressivity in the 
underlying ontology languages. In particular, beyond the straightforward \emph{intersective adjectives}, there exist \emph{gradable adjectives}, requiring fuzzy or
non-monotonic semantics, as well as \emph{operator adjectives}, requiring second-order logic for modelling. 
We consider how we can extend the ontology-lexicon interface as realized by extant models such as \emph{lemon} in the face of the issues mentioned above, in particular those arising in the context of modelling the ontological semantics of adjectives. We show how more complex logical formalisms that are required to capture the ontological semantics of adjectives can be backward 
engineered into OWL-based modelling by means of pseudo-classes. We discuss the implications of this modelling in the context of application to ontology-based question answering.
\end{abstract}



\section{Introduction}
\label{intro}
\blfootnote{  
     This work is licensed under a Creative Commons 
     Attribution 4.0 International Licence.
     Page numbers and proceedings footer are added by
     the organisers.
     Licence details:
     \url{http://creativecommons.org/licenses/by/4.0/}
}


Ontology-lexicon models, such as \emph{lemon} (Lexicon Model for Ontologies)~
\cite{mccrae2012inter} model the semantics of open class words by capturing their semantics with respect to the semantic vocabulary defined in a given ontology. Such 
ontology-lexica are built around the separation of a \emph{lexical layer} describing 
how a word or phrase acts syntactically and morphologically, and a \emph{semantic layer} 
describing how the meaning of a word is expressed in a formal logical model, 
such as OWL (Web Ontology Language)~\cite{mcguinness2004owl}. It has been 
shown that this principle known as \emph{semantics by reference}~
\cite{buitelaar2010ontology} is an effective model that can support the task of developing
question answering systems \cite{unger2011pythia} and natural language 
generation \cite{cimiano2013exploiting} over backends based on Semantic Web data models.
The Pythia system, which builds on the \emph{lemon} formalism to declaratively capture the lexicon-ontology interface, for example, has been instantiated to the case of answering questions from DBpedia \cite{unger2011pythia}.
However, as has been 
shown by the Question Answering over Linked Data \cite[QALD]{lopez2013evaluating}
benchmarking campaigns, there are many questions that can be asked over this database that require 
a deeper representation of the semantics of words, adjectives in particular. For example, 
questions such as~(\ref{ex:australia}) require understanding of the semantics of `high' in a manner that goes beyond the expressivity of OWL. The formalization of this question as an executable query formulated with respect to the SPARQL query language is provided in ~(\ref{ex:australia_query}). 
In particular, the interpretation of this question involves the formal interpretation of the word `high' as relating to the property {\tt dbo:elevation}, including ordering 
and subset selection operations.

\begin{enumerate}
\item \begin{enumerate} 
\item What is the highest mountain in Australia? \label{ex:australia}
\item \begin{small}\begin{small}\begin{verbatim}
SELECT DISTINCT ?uri WHERE { 
  ?uri rdf:type dbo:Mountain . 
  ?uri dbo:locatedInArea res:Australia . 
  ?uri dbo:elevation ?elevation . 
} ORDER BY DESC(?elevation) LIMIT 1
\end{verbatim}\end{small} \end{small}
\label{ex:australia_query}
\end{enumerate}
\end{enumerate}

It has been claimed that first-order logic and thus by extension description 
logics, such as OWL, ``fail decidedly when it comes to adjectives''
\cite{bankston2003modeling}. In fact, we largely agree that the semantics 
of many adjectives are difficult or impossible to describe in first-order logic. 
However, from the point of view of the ontology-lexicon interface, the logical 
expressivity of the ontology is not a limiting factor. In fact, due to the 
separation of the lexical and ontology layers in a model such as \emph{lemon}, 
it is possible to express the meaning of words without worrying about the 
formalism used in the ontology. To this extent, we will first demonstrate that 
adjectives are in general a case where the use of description logics (DL) breaks down, 
and for which more sophisticated logical formalisms must be applied. We then 
consider to what extent this can be handled in the context of the 
ontology-lexicon, and introduce pseudo-classes, that is OWL classes with 
annotations, which we use to express the semantics of adjectives in a manner
that would allow reasoning with fuzzy, high-order models. To this extent, we base
our models on the previously introduced design patterns~\cite{mccrae2014design}
for modelling ontology-lexica. 
Finally, we show how these semantics can be helpful in practical applications 
of question answering over the DBpedia knowledge base.

\section{Classification of adjectives}

There are a number of classifications of adjectives. First we will start 
with the most fundamental distinction between \emph{attributive} and 
\emph{predicative} usage, that is the use of adjectives in noun phrases 
(``$X$ is a $A~N$'') versus as objects of the copula (``$X$ is $A$''). 
It should be noted that there are many adjectives for which only predicative or 
attributive usage is allowed, as shown in~(\ref{ex:awake}) and~(\ref{ex:baby}).

\begin{enumerate}[resume,noitemsep]
\item \begin{enumerate}	
\item \ \,Clinton is a former president.
\item $^\ast$Clinton is former.
%\item \ \,Mary is a technical engineer. \label{ex:engineer}
%\item $^\ast$Mary is technical. (Technically this is allowed in English)
\end{enumerate}
\label{ex:technical}
\item \begin{enumerate}	
        \item \ \,The baby is awake. \label{ex:awake}
\item $^\ast$The awake baby.
\end{enumerate}
\label{ex:baby}
\end{enumerate}

One of the principle classifications of the semantics of adjectives (for example \cite{partee2003there,bouillon1999description,morzycki2013modification}) is based on the meaning of adjective noun compounds relative to the meaning of the words by themselves. This classification is as follows (where $\Rightarrow$ denotes entailment).

\begin{description}
\item[Intersective] ($X\mathrm{~is~a~}A~N \Rightarrow X\mathrm{~is~}A \wedge  X\mathrm{~is~a~}N$) 
Such adjectives work as if they were another noun and indicate that the compound 
noun phrase is a member of both the class of the noun and the class of the 
adjective. For example, in the phrase ``Belgian violinist'' it refers to a 
person in the class intersection $Belgian \sqcap Violinist(X)$, and hence we 
can infer that a ``Belgian violinist'' is a subclass of a ``Belgian''.  Furthermore,
we could conclude that if the same person were a surgeon, he/she would also
be a ``Belgian Surgeon''.

\item[Subsective] ($X\mathrm{~is~a~}A~N \Rightarrow X\mathrm{~is~a~}N$, but $X\mathrm{~is~a~}A~N \not\Rightarrow X\mathrm{~is~}A$) Such adjectives acquire their specific meaning in combination with the noun the modify. For example, a ``skilful 
violinist'' is certainly in the class $Violinist(X)$ but the described person is `skilful as a violinist', but not skilful in general, e.g. as a surgeon.

\item[Privative] ($X\mathrm{~is~a~}A~N \not\Rightarrow X\mathrm{~is~a~}N$) 
These adjectives modify the meaning of a noun phrase to create a noun phrase 
that is potentially incompatible with the original meaning. For example, a 
``fake gun'' is not a member of the class of guns.
\end{description}

%This classification is useful, however one further case is important to 
%distinguish and that is of \emph{relational} adjectives\footnote{Which mean in the sense of 
%    expressing a relatioship, as opposed to in the sense of McNally and Boleda~\shortcite{mcnally2004relational}}
%    which have a meaning 
%that expresses a relationship between two individuals or events, for example:
%
%\begin{quote}
%
%He is related to her.
%\end{quote}

Another important distinction is whether adjectives are 
\emph{gradable}, i.e. whether a comparative 
or superlative statement with these adjectives makes sense. For example, adjectives such as 
`big' or `tall' can express relationships such as `$X$ is bigger than $Y$'. 
However it is not possible to say that one individual is `more former'. Most gradable 
adjectives are subsective (e.\,g.`a big mouse' is not `a big animal' \cite{morzycki2013nonscales}). 
%An 
%important group of gradable adjectives are intersective we call 
%such adjectives `absolute' (following \cite{rusiecki1985adjectives}) as they 
%refer to an ideal point on some scale. For example, a `straight line' is `straight' in 
%that it has no bends or kinks. However, we can still talk about a line being 
%`straighter' in the sense of closer to the ideal of straightness than some other object

Finally, we consider \emph{operator} or \emph{property-modifying} adjectives. 
They can be considered the same as privative adjectives, but in this 
case they are understood as operators that change some property in the qualia 
structure of the class. For instance, we may express the adjective `former' 
in lambda calculus as a function that takes a class $C$ as input and returns the class 
of entities that were a member of $C$ to some prior time point $t$~\cite{partee2003there}:

\vspace{-1.0em}
$$\lambda C [\lambda x \exists t C(x,t) \cap t < \mathrm{now}]$$
\vspace{-1.5em}

Such adjectives have not only a difference in semantic meaning but can also 
frequently have syntactic impact, for example in adjective ordering 
restrictions, as they may be reordered with only semantic 
impact~\cite{teodorescu2006adjective}, e.g.,

\begin{enumerate}[resume,noitemsep,nolistsep]
\item \begin{enumerate}
\item \ A big red car.
\item $^?$A red big car.
\end{enumerate} 
\label{ex:car}
\item \begin{enumerate}
\item A famous former actor.
\item A former famous actor.
\end{enumerate}
\label{ex:actor}
\end{enumerate}

Finally, we define \emph{object-relational} adjectives as those adjectives which have a meaning 
that expresses a relationship between two individuals or events\footnote{Our definition of relational here is borrowed from the idea of
    relational nouns~\cite{de1988interpretation} as a word that requires an argument. Our definition is also different from the one for `relational adjectives' as proposed by \cite{morzycki2013nonscales}.}, for example:

\begin{enumerate}[resume,noitemsep,nolistsep]
\item He is related to her.
\item She is similar to her brother. 
\item This is useful for something. 
\end{enumerate}

%Another important distinction to make with adjectives, which is orthogonal to the above classification, 
%is whether they are \emph{gradable}, in that whether it makes semantic senses to make a comparative 
%or superlative statement with these adjectives. For example, adjectives such as 
%`big' or `tall' can express relationships such as `$X$ is bigger than $Y$', 
%however it is not possible to say one individual is `more former'. It should thus be noted that most gradable adjectives are mostly intersective,
%for example a `big mouse' is not a `big animal' \cite{morzycki2013modification}.
%An important group of gradable adjectives are, however, intersective, and we call 
%such adjectives \emph{absolute} (following \cite{rusiecki1985adjectives}) as they 
%refer to an ideal point on some scale, such as `straight', and frequently refer
%to an endpoint of a scale, which Kennedy~\shortcite{kennedy1999scalar} calls
%a \emph{trivial standard}.

\section{Representation of adjectives in the ontology-lexicon interface}

In general it is assumed that adjectives form frames with exactly one argument 
except for extra arguments provided by adjuncts, typically prepositional phrases. 
Most adjectives are thus associated with a predicative frame, which much
like the standard noun predicate frame ($X\text{ is a }N$) is stereotyped in English as:

\vspace{-1.0em}
$$X\mathrm{~is~}A$$
\vspace{-1.5em}

The attributive usage of an adjective is associate to a stereotypical frame where the $N?$ argument is not semantically bound, but can instead be obtained by syntactic unification to a noun predicate frame:


\vspace{-1.0em}
$$X\text{ is }A~N?$$
\vspace{-1.5em}

As such, when we encounter the attributive usage of an adjective such as in~\ref{ex:juan}, 
we understand this as the realization of two frames, given in~\ref{ex:juan_frames}.

\begin{enumerate}[resume,noitemsep,nolistsep]
\item Juan is a Spanish researcher. \label{ex:juan}
\item \begin{enumerate}
\item Juan is a researcher.
\item Juan is a Spanish $N?$
\end{enumerate}
\label{ex:juan_frames}
\end{enumerate}

Note that we do not provide modelling for adjectives where the meaning is unique for a particular noun phrase,
such as `polar bear', which we would capture as a normal noun phrase with 
meaning \emph{ursus maritimus}.

\subsection{Intersective adjectives} \label{sec:intersectives}

Intersective adjectives are the most straightforward class, as in many cases they 
can be modelled essentially as a noun or verb (e.g. deverbal adjectives such as `broken'). Intersective 
adjectives take one argument and can thus be modelled as unary predicates in first-order logic or classes in OWL, as described by M\textsuperscript{c}Crae and Unger \shortcite{mccrae2014design}. For practical modelling examples, we will use the
\emph{lemon} model, since it is the most prominent implementation of the 
ontology-lexicon interface.

The primary mechanism of modelling the syntax-semantics interface in the context 
of \emph{lemon} is by means of assigning a \emph{frame} as a \emph{syntactic 
behaviour} of an entry and giving it \emph{syntactic arguments}, which can then 
be linked to the \emph{lexical sense}, which stands proxy for a true semantic 
frame in the ontology. For example, the modelling of an adjective such as 
`Belgian' can be achieved as follows (depicted in Figure 
\ref{example-belgian})\footnote{We assume that the namespaces are defined for 
the lexicon as {\tt lexicon}, e.g., \url{http://www.example.org/lexicon}
and for the entry, e.g., {\tt belgian} is \url{http://www.example.org/lexicon/belgian#}.
Other namespaces are assumed to be as usual.}.

\begin{figure}
\includegraphics[width=\textwidth]{belgian-example}
\caption{Modelling of an intersective adjective `Belgian' in \emph{lemon}\label{example-belgian}}
\end{figure}

\begin{small}\begin{verbatim}
lexicon:belgian a lemon:LexicalEntry ;
  lemon:canonicalForm belgian:Lemma ;
  lemon:synBehavior   belgian:AttrFrame , 
                      belgian:PredFrame ;
  lemon:sense         belgian:Sense .

belgian:Lemma lemon:writtenRep "Belgian"@eng .

belgian:AttrFrame lexinfo:attributiveArg belgian:AttrSynArg .
belgian:PredFrame lexinfo:copulativeArg  belgian:PredSynArg .

belgian:sense lemon:reference [ a owl:Restriction ;
                                owl:onProperty dbpedia:nationality ;
                                owl:hasValue dbpedia:Belgium ] ;
              lemon:isA belgian:AttrSynArg , belgian:PredSynArg .
\end{verbatim}\end{small}

Note that here we use the external vocabulary defined in the LexInfo ontology \cite{cimiano2011lexinfo} 
to define the meaning of the arguments of the frame as the \emph{attributive 
argument}, corresponding to the frame stereotype `$X\text{ is }A~N?$' and the 
\emph{copulative argument} for the frame stereotype `$X\mathrm{~is~}A$'. Furthermore,
the class of Belgians is not named in our reference ontology DBpedia, so we 
introduce an anonymous class with the axiomatization, 
$\exists\,\text{\em nationality}\,.\,\text{\em Belgium}$. It is in fact common that the 
referent of an adjective is not named in an ontology. An obvious choice is thus to model denominal adjectives as classes of the form $\exists\,\text{\em prop}.\text{\em Value}$, 
where $\text{\em Value}$ is an individual that represents the semantics of the noun from which the adjective was derived. This modelling is so common that it has already been encoded as two
design patterns, called {\tt IntersectiveObjectPropertyAdjective} and {\tt
IntersectiveDatatypePropertyAdjective} (see \cite{mccrae2014design}).
Similarly, most deverbal adjectives refer to an event, and as such
a common modelling is of the form $\exists\,\text{\em theme}^{-1}. \,\text{\em EventClass}$.
For example, `vandalized' may be $\exists\,\text{\em theme}^{-1}.\text{\em VandalismEvent}$.

\subsection{Gradable adjectives and relevant observables} \label{sec:gradables}

Gradable adjectives have a number of properties, which differentiate them
from intersective adjectives:

\begin{itemize}[noitemsep]
\item They occur in comparative constructions, in English with either `-er' or `more' \cite{kennedy1999scalar}, e.g. `smaller' and `more frequent', as opposed to intersectives such as `*less geological' and `*more wooden'.
\item Gradable adjectives can be defined as `scalar', since their value can ideally be measured on a scale of set degrees
%, which is dependent on the object of the adjective (`big' can be defined based on the number of stores of a `building' or the height of a `person').
\item They have a context-dependent truth-conditional variability, meaning that their positive form
    is understood in relations to the class of the object modified by the adjective. For example, an
    `expensive watch' has a different price scale to an `expensive bottle of water'.
%is the sum of the relation between the degree of the concept possessed by the object (as measured by the predicate) and the context-dependent standard of comparison based on the same concept \cite{kennedy2007vagueness}. It follows that the properties denoted by adjectives like `expensive' or `small' or `big' vary in intensity according to the context (and time) of use.} \textcolor{blue}{This part was considered not to be clear by Philipp, he comments: ``Examples would help here definitely, e.g. for what we mean with standard of comparison, intensity, sum of relation between the degree and context-depenedent standard etc. These notions are hard to grasp.''. We could rephrase it by saying: Gradable adjectives have a context-dependent truth-conditional variability. This means that the value of the adjective depends (and is gradable) according to the object it qualifies \cite{kennedy2007vagueness}. Adjectives such `big', `expensive' or `small' are given different values according to the objects they are associated with.}} 
%\item \textbf{\textcolor{violet}{The arguments of gradable adjectives are mapped into abstract representations of measurements or degrees~\cite{kennedy2007vagueness}.} \textcolor{blue}{This part was considered not clear by Philipp. Possible rephrase: It follows that 
\item They are frequently \emph{fuzzy} (or \emph{vague})~\cite{kennedy2007vagueness}.
%\item \textbf{\textcolor{blue}{\textit{absolute} gradable adjectives, like `straight' or `bent' or `red' (to cite your part on colors below), are not vague}}

\item There may be a minimum or maximum of the adjective's scale, which can be determined
by, for example, whether they can modified by adverbs such as `completely' or `utterly'.
\end{itemize}

As such, we define gradable adjectives relative to a particular 
property. These adjectives are also called `observable'~\cite{Bennett06kr}\footnote{Note that in many cases the property is quite abstract such as in 
`breakable'.} as they are related to some observable or measurable property, e.g. \emph{size} in the case of \emph{`big'}. However, a specification of the observable property is clearly not sufficient to differentiate between the meaning of antonyms such as \emph{big} and \emph{small.}
Thus, we introduce the notions of \emph{covariance} and \emph{contravariance}, which specify 
whether the comparative form indicates a higher property value for the subject 
or the object. In this sense `big' is covariant with size, as bigger things have 
a higher size value, and `small' is contravariant with size.\footnote{
    The use of these terms is borrowed from type systems, and resembles the concept of `converse observables' as introduced by (\cite{Bennett06kr}:42). As stated by the author, adjectives often come in pairs of polar opposites (e.\,g. $\text{\em conv}(\text{\em tall})=\text{\em short}$, and both refer to the same observable (in this case \textit{size}). Some observables analogously hold converse relationships with other observables (e.\,g. $\text{\em conv}(\text{\em flexibility})=\text{\em rigidity}$ or $\text{\em conv}(\text{\em tallness})=\text{\em shortness}$).} 
    We also introduce 
a third concept of \emph{absolute gradability}, which states that these objects 
are better described by these adjectives as they approach some ideal value. 
A common example of this is colours, where we may say that some object is 
redder than another if it is closer to some ideal value of red 
(e.g., RGB {\tt 0xff0000}).

While these notions can handle the comparative structure of the semantics of adjectives, the 
predicative and superlative usage of adjectives is complicated by three factors
that we will outline below. We notice that gradable classes are not 
crisply defined like in the case of many intersective adjectives. In fact, while we can clearly define all people in the world as `Belgian' or 
`not Belgian', according to whom holds a Belgian passport or not, it is not easy to
split the world's population into `tall' and `not tall' (This is known as \emph{sorites} paradox \cite{Bennett06kr}).
%, for which many terms and adjectives in natural language are vague given the 
%existence of a specific threshold to relate them to other properties or observables.}} 
Furthermore, while it may 
be easy to say that someone with height 6'6'' (198cm) is `tall', it is not clear 
whether someone with height 6' (182cm) is `tall', although compared to an average (different)
height for a man, they are `taller'.
As such, one frequently used way to deal with this class of vague adjectives (and nouns) is via fuzzy logic \cite{Goguen1969fuzzy,Zadeh1975fuzzy,Zadeh1965fuzzy,DuboisPrade88en,Bennett06kr}.
Secondly, we notice that these class boundaries are 
non-monotonic, that is that with knowledge of more instances of the relative 
class we must revise our class boundaries. This is especially the case for superlatives, as the discovery of a new tallest person in the world would remove 
the existing tallest person in the world from the class of tallest person in the 
world. This non-monotonicity also affects the class boundaries of the gradable 
class itself. For example, in the 18th century, the average height of a male was 
5'5'' (165cm)\footnote{\url{https://en.wikipedia.org/wiki/Human_height}}; as such a male of 6' would have clearly been considered tall. 

It follows that each instance added to our ontology might lead to a revision of the class boundaries of a gradable class, hence leading to the fact that 
gradable adjectives are fundamentally non-monotonic. We must also notice 
that gradability can only be understood relative to the class that we wish to 
grade, that is that while it is unclear as to whether 6' is tall for a male, 
given the current average height of a female being about 5'4'' (162cm), it is clear 
that 6' is tall for a female.

We can therefore conclude that gradable adjectives are \emph{fuzzy}, \emph{non-monotonic} 
and \emph{context-sensitive}, all of which are incompatible with the description 
logic used in OWL.

\subsubsection{Pseudo-classes in lemonOILS}

Currently there are only limited models for representing fuzzy 
logic in the context of the Web~\cite{zhao2008uncertainty}. In order to capture the 
properties of gradable adjectives, we introduce a new model, which we name 
\emph{lemonOILS} (The \emph{lemon} Ontology for the Interpretation of Lexical Semantics)\footnote{\url{http://lemon-model.net/oils}}. This ontology introduces three 
new classes:

\begin{itemize}[noitemsep]
	\item {\tt CovariantScalar}, indicating that the adjective is covariant with its bound property
	\item {\tt ContravariantScalar}, indicating that the adjective is contravariant with its bound property
	\item {\tt AbsoluteScalar}, indicating that the property represents similarity to an absolute value
\end{itemize}

In addition, the following properties are introduced to enable the description 
of gradable adjectives. Note that all these properties are typed as 
\emph{annotation properties} in the OWL ontology, so that they do not interfere 
with the standard OWL reasoning.

\begin{itemize}[noitemsep]
	\item {\tt boundTo} indicates the property that a scalar refers to (e.g., `size' for `big')
	\item {\tt threshold} specifies a sensible minimal value for which the adjective can be said to hold
        \item {\tt absoluteValue} is the ideal value of an absolute scalar
	\item {\tt degree} is specified as {\tt weak}, {\tt medium}, {\tt strong} or {\tt very strong}, corresponding to approximately 50\%, 25\%, 5\% or 1\% of all known individuals
	\item {\tt comparator} indicates an object property that is equivalent to the comparison of the adjective (e.g., an object property {\tt biggerThan} may be considered a comparator for the adjective class {\tt big})
	\item {\tt measure} indicates a unit that can be used as a measure for this adjective, e.g., `John is 175 \textit{centimetres} tall'.
\end{itemize}

Using such classes we can capture the semantics of gradable adjectives 
syntactically but not formally within an OWL model. As such, we call these 
introduced classes \emph{pseudo-classes}. An example of modelling an adjective 
such as `high' is given below (and depicted in Figure \ref{high-example}).

\begin{figure}
\includegraphics[width=\textwidth]{high-example}
\caption{An example of the modelling of `high' in \emph{lemon}\label{high-example}}
\end{figure}

\begin{small}\begin{verbatim}
lexicon:high a lemon:LexicalEntry ;
  lemon:canonicalForm high:Lemma ;
  lemon:synBehavior high:PredFrame ;
  lemon:sense high:Sense .

high:Lemma lemon:writtenRep "high"@eng .

high:PredFrame lexinfo:copulativeArg high:PredArg .

high:Sense lemon:reference [
    rdfs:subClassOf oils:CovariantScalar ;
    oils:boundTo dbpedia:elevation ;
    oils:degree oils:strong ] ;
  lemon:isA high:PredArg .
\end{verbatim}\end{small}

%\textbf{\textcolor{blue}{Philipp's comment}\textcolor{violet}{3.2.1. it becomes now clear that Markov Logic is one way of implementing the oils-based description.
%Still, it would be nice to provide some alternatives, grounded in the DBpedia data. Another way would be to construct a distribution of the top 5\% (oils:strong) elements according to size for different classes and show the distributions graphically. Fuzzy sets would essentially be similar as they are essentially approximations of distributions without normalization I would say.
%One problem I see with the modelling in 3.2.1 of high we do not model the actual reference class, i.e. the noun. That's fine, but it is worth mentioning that the semantics (at least the degree and contravariance/variance) is independent of the specific noun (e.g. whether we talk about mountains or mice), but the threshold for instance and defaultValue are surely going to be specific for the noun/class modified. This needs to be pointed out IMHO.}}


As an example of a logic in which these 
annotations could be interpreted, we consider Markov Logic~\cite{richardson2006markov}, which is an 
extension of first-order logic, in which each clause is given a cost. The process 
of reasoning is thus transformed into an optimization problem of finding the 
extension, which minimizes the summed weight of all violated clauses. As such, we
can formulate a gradable adjective based on the number of known instances. 
For example, we can specify `big' with respect to \emph{size} for some class $C$ as in~(\ref{ex:big}).
%
\begin{enumerate}[resume,noitemsep]
\item $\forall x \in C, y \in C : size(x) > size(y) \rightarrow big_C(x) : \alpha$ \\
$\forall x \in C, y \in C : size(x) < size(y) \rightarrow \neg big_C(x) : \beta$
\label{ex:big}
\end{enumerate}
%
In this way, the classification of an object into `big' or `small' can be defined as follows.
For an individual $x \in C$, the property $big_C(x)$ holds if and only if: 

\vspace{-1.0em}
$$|\{y \in C, size(y) > size(x)\}| \alpha < |\{y \in C, size(y) < size(x)\}| \beta$$
\vspace{-1.5em}

where the values of $\alpha$ and $\beta$ are related to the degree defined
in the ontology.

We see that `big' defined in this way has the three properties outlined above: 
it is non-monotonic (in that more individuals may change whether we consider an individual 
to be `big' or not), it is fuzzy (given by the strength of the probability of the proposition $big_C(x)$), 
and it is context-sensitive (as whether an individual counts as big or not depends on the class $C$). 
Furthermore, our definition does not rely on defining `big' for a given class, but instead is inferred
from some known number of instances of this class. This eliminates the need to
define a threshold for each individual class, or even to define the predicate $big_C$ on a per-class
basis.

\subsubsection{The supervaluation theory and SUMO}
%%%%% John: these criticisms don't really hold for Markov Logic, I think I remove them
%\textbf{\textcolor{blue}{Despite the successful application of fuzzy logics in the case of lemonOILS, we acknowledge at least two bottlenecks in the use of this kind of logic for modelling vague or gradable adjectives.}}
%
%\begin{itemize}[noitemsep]
%\item \textbf{\textcolor{blue}{the values which are assigned to the thresholds or degrees to which an object satisfies a vague predicate are assigned \textit{ad hoc}, meaning that there is no objectivity in appointing them.}}
%\item \textbf{\textcolor{blue}{Fuzzy logic is inferentially weak. Its fixed, domain-independent operators do not extensively work for uncertain information as in the case of vague predicates and observables, and could lead to unsatisfied or incorrect inferences (\cite{Elkan94fuzzy2}:5); \cite{Elkan93fuzzy1}.  Given for instance a vague piece of information such as `The mountain is far from the sea; and my house is beside the mountain'' (\cite{Bennett06kr}:36), one can automatically conclude `My house is far from the sea', a kind of inference that fuzzy logic cannot provide. \textcolor{violet}{[John, it would be useful to try to represent this with fuzzy logic operators]}}}
%\end{itemize}


Another way to capture the meaning of these vague terms can be achieved by \textit{supervaluation semantics}. 
Through supervaluation theory, the modelling or positioning of \emph{sorites} vague concepts is grounded in a 
judgement or meaning that lies on arbitrary thresholds, but these thresholds are based on a number of \textit{relevant objective measures} \cite{Bennett06kr}. 
%The principle implies that the thresholds used for semantically interdependent concepts 
%(let them be predicates or observables) are used consistently (e.\,g. the thresholds for `tall/tallness' 
%closely resemble the thresholds for `short/shortness'). It also implies that not only the meaning of 
%the gradables, but also the set value of the threshold is context-sensitive.
 
%In the case of (as inspired by (\cite{kennedy1999scalar}:129)):

%\begin{enumerate}[resume,noitemsep,nolistsep]
%\item \textbf{\textcolor{blue}{Michael Jordan is tall. \label{ex:jordan}}}
%\begin{enumerate}[label=$\diamond$]
%\item \textbf{\textcolor{blue}{The threshold for `height' can either be set (1) with reference to the average height for baseball players, or (2) with reference to the average height for Human, namely Person (where the capital starting letter of the word stands for ontological concept or KB term).}}
%\end{enumerate}
%\item \textbf{\textcolor{blue}{The building is tall. \label{ex:building}}}
%\begin{enumerate}[label=$\diamond$$\diamond$]
%\item \textbf{\textcolor{blue}{The threshold for `height' can either be set with reference to the average height of a specific kind of Building, or with reference to a specific building.}}
%\end{enumerate}
%\end{enumerate}

%\begin{figure}[ht!]
%\includegraphics[width=\textwidth]{building_SUMO1}
%\caption{Excerpt of the SUMO graph structure for \texttt{Building}, with number of direct children for each entry and related documentation}\label{building_SUMO}
%\end{figure}


A recent extension of the SUMO ontology~\cite[Suggested Upper Merged Ontology]{nilespease2001sumo}\footnote{\url{www.ontologyportal.org}} includes
default measurements (currently amounting to 300+) added to the \texttt{Artifact}s, \texttt{Device}s and \texttt{Object}s enlisted in the ontology (and marked with capitals).
%mentioned in $\diamond$(2) and $\diamond$$\diamond$ have already been contemplated in one recent extension of the SUMO ontology\footnote{The Suggested Upper Merged Ontology (SUMO); \cite{nilespease2001sumo}; \url{www.ontologyportal.com}}, namely the addition of defaultMeasurements for the Artifacts, Devices and Objects enlisted in the ontology (currently amounting to 300+). The SUMO ontology can be browsed either via English terms (as derived from the Princeton WordNet \textregistered), or KB terms, where the upper-level concepts have been converted in the formal first-order logic language SUO-KIF.}}
The compilation of \texttt{defaultMeasurements} in SUMO has been just conducted on observables, not on predicates. Given for instance an \texttt{Artifact} such as \texttt{Book}, the compilation of its default measurements would look like:

\begin{small}\begin{verbatim}
;;Book
(defaultMinimumHeight Book (MeasureFn 10 Inch))
(defaultMaximumHeight Book (MeasureFn 11 Inch))
(defaultMinimumLength Book (MeasureFn 5.5 Inch))
(defaultMaximumLength Book (MeasureFn 7 Inch))
(defaultMinimumWidth Book (MeasureFn 1.2 Inch))
(defaultMaximumWidth Book (MeasureFn 5.5 Inch))
\end{verbatim}\end{small}

The example for \texttt{Book} shows that the default measurements for the observable reflect a \textit{standard} kind of book,
i.e., one of the most commonly known kinds of the same artifact. As for this case, SUMO implies \texttt{Book} to be a physical object with a certain length, height and width (and possibly weight). 
A weakness here is that the there is no systematic connection between the {\tt defaultMinimumHeight} and {\tt Height} or {\tt Width}, since these physical properties have been defined in SUMO just in terms of first-order logic, and have not been assigned default measurements yet. With \emph{lemonOILS} we can add this information as follows:

\begin{small}\begin{verbatim}
sumo:Book oils:default [
    oils:defaultFor sumo:height ;
    oils:defaultMin "10in" ;
    oils:defaultMax "11in" ] .
\end{verbatim}\end{small}

Then, if we understand a lexical entry `high' as referring to a scalar covariant pseudo-class for {\tt sumo:height},
it is possible to understand that a `high' object exceeds the default minimum set established for the same object and owns at the same time a value for `high' which does not go beyond the established default maximum. %The decision to set an arbitrary threshold for objective measures to every specific kind of \texttt{Artifact} enlisted in SUMO presents advantages and disadvantages or bottlenecks. One obvious advantage is that SUMO poses itself, with this further extension on default physical measurements, as the first ontology of its kind to present defaults in correlation with its \texttt{Artifact}s. These approximated values, which ideally should represent the most commonly known form of each of the \texttt{Artifact}, can become handy in need of measurements or further information on the \texttt{Artifact}s. The approximated values also enable to draw a comparison between the children of the same parent. In this way, it might be possible, given for instance the defaults for \texttt{Bed}, to approximate the defaults for \texttt{KingBed}, \texttt{SingleBed}, \texttt{SofaBed}, \texttt{Futon} or \texttt{DoubleBed}, simply by adding or subtracting height, width, length and weight values with \texttt{Bed} as threshold.
A further weakness of this approach is captured by the following example:

\begin{enumerate}[resume,noitemsep]
    \item Avery Johnson is a short basketball player. \label{ex:avery}
%\item Michael Jordan is tall. \label{ex:jordan}
%\begin{enumerate}[label=$\diamond$]
%\item \textbf{\textcolor{blue}{The threshold for `height' can either be set (1) with reference to the average height for baseball players, or (2) with reference to the average height for Human, namely Person (where the capital starting letter of the word stands for ontological concept or KB term).}}
%\end{enumerate}
%\item \textbf{\textcolor{blue}{The building is tall. \label{ex:building}}}
%\begin{enumerate}[label=$\diamond$$\diamond$]
%\item \textbf{\textcolor{blue}{The threshold for `height' can either be set with reference to the average height of a specific kind of Building, or with reference to a specific building.}}
%\end{enumerate}
\end{enumerate}

Here, we see the difficulty in interpreting the sentence, as Avery Johnson is in fact of average height (5'10'') but
for the class of basketball players he is unusually short. While SUMO has some very specific listings of subsets for the same \texttt{Artifact}\footnote{For example, some of the subsets \texttt{Car} are: {\tt CrewDormCar}, {\tt GalleryCar}, {\tt MotorRailcar}, {\tt FreightCar}, {\tt BoxCar}, {\tt RefrigeratorCar}, {\tt FiveWellStackCar}, and more.}, %for which it is then relatively simple to define maximum and minimum defaults for each for them and therefore consequently readjust the defaults for \texttt{Car} so to be the most prototypical possible, in some other places of its conceptual taxonomy, the ontology is still behind and needs further extension. For instance, with reference to this example of the basketball player, 
SUMO does not provide a well-structured subset net for e.\,g. \texttt{Person}%, which is just enlisted as an `instance for \texttt{Human}'. Further classifications of \texttt{Person} (e.\,g. in terms of gender or age) are still missing.
. A way to solve this bottleneck could be to introduce, for every possible subclass of {\tt Person}, default values; as well as to introduce default values for the same Artifact in conjunction with a predicate or adjective (e.\,g. {\tt BigPerson}, {\tt BulkyPerson}). Nevertheless, this can potentially lead to an explosion in the amount of information we need to encode in the ontology.
%, essentially forcing us to encode a class (or default values) for every
%combination of an adjective and a noun. 
 On the other side though, the SUMO default measurements serve the purpose they were originally conceived for, namely to be an arbitrary, yet computable approximation of physical measures.

%%The observable (hereby as $c$) is therefore treated not as a classical observable (unique in its kind), but rather as an instance of the same ($x$), to which specific (implied) default values are assigned (\textbf{$a$}): \[ [\textbf{a} \circ c](x)\]also representable as \[ \texttt{lsa}(x, \textbf{a}, c).\]}
%
%
%\textbf{\textcolor{blue}{SUMO's pros}}
%\begin{itemize}[noitemsep][label=$\bar$]
%\done \textbf{\textcolor{blue}{The compilation of default measurements limits the extent of first-order semantics that one could infer from SUMO.}}
%\done bla bla 
%\end{itemize}
%
%
%
%
%\textbf{\textcolor{blue}{SUMO's cons}}
%\begin{itemize}[noitemsep]
%\crossed \textbf{\textcolor{blue}{Almost all children of the enlisted Artifacts in SUMO could be assigned default values, but default values for the parents (e.\,g. `Artifact', `OrganicObject', `Object') could not be defined being too vague. Also, while some Artifacts have extensively defined children (e.\,g. `CrewDormCar', `GalleryCar', `MotorRailcar', `FreightCar', `BoxCar', `RefrigeratorCar', `FiveWellStackCar', `FlatCar', `SpineCar', `HydraCushionFreightCar' and more just for `Car'), some Artifacts are still unspecified (e.\,g. `Person', as Instance for `Human', does not contain any disambiguation in terms of i.\,a. gender or age).}}
%
%\crossed \textcolor{blue}{Complex count nouns such as `TallBigMan' or `SmallRoundWoman' have not been contemplated among the defined valued Objects yet. The extension of these subsets (defined restrictive adjectives by \cite{Bennett06kr}) should nevertheless be a feasible task to accomplish, provided that these predicates belong to the accepted collocational\footnote{One condition for the predicates of the observable / observable's instance to be analyzed is that they count as collocational forms once merged to it in a subset. In other words, while it can be feasible to set default values for a subset such as `BroadShoulder' (since the adjective `broad' standardly collocates with the noun `shoulder'), the same cannot be said for e.\.g. `BroadWoman', where the adj.+noun matching is novel or unconventional. A stand-alone case is also represented by idiomatic or metaphorical expressions, such as `high risk' or `to keep one's head up high'.} cluster of the observable and of its instance. This extension should therefore include all the cases in which the default value \textbf{$a$} is not implied, but explicited ($\texttt{lsa}(x, \textbf{a}, c)$, with \textbf{$a$} being `big' as for instance in the case of `BigMan').}
%\end{itemize}
%
%
%\textbf{Thresholds, defaults and multiple classes (meaning context-sensitivity but also predicate+observable, e.\,g. should the ontology have a different class for BigMan, BigWoman, BigBasketballPlayer..) (Francesca to help)}

\subsection{Operator adjectives} \label{sec:operators}

Operator adjectives are those that combine with a noun to modify the meaning of the noun itself. 
There are two primary issues with the understanding of the adjective in this manner. 
Firstly, the reference of the lexical item does not generally refer to an existing item 
in the ontology, but rather is novel and productive, in the sense that it generates a new class. Secondly, the compositional nature 
of adjective-noun compounds is no longer simple, as in the cases of intersective and gradable adjectives.
This means that, in order to understand a concept such as a `fake gun', we must first
derive a class of {\tt FakeGun}s from the class of {\tt Gun}s. Thus the modified noun phrase
must be an argument of the operator adjective.

%\begin{enumerate}[resume,noitemsep,nolistsep]
%\item \textbf{\textcolor{blue}{Clinton is a male tall former president}} \label{ex:clinton}
%\item \textbf{\textcolor{blue}{$ |= $ Clinton is president.}}
%\item \textbf{\textcolor{blue}{$ |= $ Clinton is male (male $\Rightarrow$ intersective gradable adjective)}}
%\item \textbf{\textcolor{blue}{$ |= $ Clinton is tall (tall $ \Rightarrow$ subsective gradable adjective)}}
%\item \textbf{\textcolor{blue}{$\emptyset$ Clinton is former.}} \textbf{\textcolor{blue}{(Since this example was the same as (2a,b) above, I've changed 2(a)(b) into `Mary is a technical engineer $\Rightarrow$ *Mary is technical)}}
%\end{enumerate}

To this extent we claim that it is not generally possibly to represent the 
meaning of an operator adjective within the context of an OWL ontology.
Instead, following Bankston~\cite{bankston2003modeling}, we claim that
the reference of an operator adjective must be a higher order predicate.
If we assume that there are operators of the form of a function, then
the argument of a operator is the attributed noun phrase. As such, we introduce a frame \emph{operator attributive}, that has one argument
which is the noun. Thus we understand that the interpretation of `fake gun' is
by means of an operator $fake$, which is a function that takes a class 
and produces a new class, i.e., $[fake(Gun)](X)$. It is of course not possible
in first-order logic to have such a function, but is in higher-order
logic.
%
%The interpretation of a phrase like~(\ref{ex:president1}) is like~(\ref{ex:president2}).
%
%\begin{enumerate}[resume,noitemsep]
%\item \begin{enumerate}
%\item Obama is a president. Clinton is a former president. \label{ex:president1}
%\item $\text{\em President}(\text{\em Obama}) \sqcap [\text{\em former}(\text{\em President})](\text{\em Clinton})$ \label{ex:president2}
%\end{enumerate}
%\end{enumerate}
%
%If we understand {\em President} as \texttt{Artifact} or \texttt{Human} that has the role of leading a country
%(and therefore being true for {\em Obama} and {\em Clinton}), then {\em former} creates a new class in the ontology (i.e. \texttt{FormerPresident}) which selects
%people who were presidents.  An alternative
%    interpretation would be that `former' modifies the role of Clinton, however this raises the
%    question of why the role property of presidents would be selected, and as such
%is also unsatisfactory. %{\textbf{\textcolor{blue}{In a way, we have already answered this statement by means of the grammatical definition that we have given for relational and operator adjectives, namely that they cannot take a postnominal position (unlike gradable adjectives). It follows that we can avoid discussing the case that `former' modifies the role of Clinton, given that the statement `Clinton is former' is \textit{de facto} ungrammatical. Not only this, but the adjective `former' (as other operator adjectives such as `alleged' or `counterfeit') is neither intersective, nor subsective, but privative \cite{partee2001privative}, meaning that it \textit{entails the negation of the noun property}. Therefore, to answer the question of why the role property of presidents would be selected, we can state that the operator adjective raised the role of the object, but dismissed it at the same time.}}
%As there is at the moment no agreed way of representing the semantics of such an operator in an ontology
%for such an operator in an
%ontology, it underlines the fact that more sophisticated ontology representations than first-order logic are %required to understood natural
%language in general.
To fully capture the semantics of such an operator adjective, formalisms beyond first-order logic are thus clearly needed.

%This means that we must acknowledge operator adjectives in both the lexicon and the ontology. 
%To this extent we define a frame called operator adjective frame, whose prototype is:
%
%$$X\mathrm{~is~a~}A~NP$$
%
%This leads to the odd case that operator adjectives are then considered the 
%head of the frame! \textbf{hmm...} In this case 
%we can understand the reference of the adjective as a property that relates an 
%individual to a class. As such it is clear, that the reference of an operator 
%adjective is a higher-order predicate. Fortunately, in the case of OWL we can 
%cheat on this second-order nature by means of \emph{punning}, which allows a 
%class to also be an individual. If we thus assume that operator adjectives are 
%essentially puns, then it follows that we can assume that the reference of an 
%operator adjective is thus a property. As such, for example, we can model an 
%adjective such as `former' as referring to a property such as {\tt heldRole} 
%whose range is a class of roles punned as individuals. This approach is 
%effective, however it has limits in general \textbf{does it???}
%
%We will not be able to easily create a vocabulary that can fully describe the 
%semantics of the adjective within the context of OWL as the second nature order 
%of the logic cannot be captured well within the framework of description logic. 
%However, we can use the punning trick described above to capture the semantics
%of the adjective. To do this we need to add a frame on the syntax side, that 
%indicates that the argument of the adjective is in fact the noun phrase. We 
%would do this as follows (see also Figure \ref{former-example}):
%
%\begin{figure}
%\includegraphics[width=\textwidth]{former-example}
%\caption{An example of modelling `former' in %\emph{lemon}\label{former-example}}
%\end{figure}
%
%\begin{small}\begin{verbatim}
%former: a lemon:LexicalEntry ;
%	lemon:canonicalForm former:Lemma ;
%	lemon:synBehavior former:OperatorFrame ;
%	lemon:sense former:Sense .
%
%former:Lemma lemon:writtenRep "former"@eng .
%
%former:OperatorFrame lexinfo:copulativeArg former:Subject ;
%  lexinfo:attributiveOperator former:Object .
%  
%former:Sense lemon:reference onto:heldRole ;
%  lemon:subjOfProp former:Subject ;
%  lemon:objOfProp former:Object .
%\end{verbatim}\end{small}
%
%The usage of this frame is intended such that a phrase such as:
%
%\begin{quote}
%Clinton is a former president
%\end{quote}
%
%Is interpreted as:
%
%\begin{small}\begin{verbatim}
%ontology:Clinton ontology:heldRole ontology:President .
%\end{verbatim}\end{small}


%\textbf{\textcolor{blue}{Philipp's original comment (prior to my change):}\textcolor{violet}{On operational adjectives and the specific example `Clinton is a former president'.
%I have problems with this example, I see the class of president as containing a set of individuals representing a role played by someone beginning and ending at some time point.
%So former would be a class that out of these role individuals selects all those with ending point before now. This seems quite doable in FOL to me ;-)}}

\subsection{Object-relational adjectives}

Object-relational adjectives are those that require a second argument, such as `known', which
can only be understood as being `known' to some person, in comparison to `famous'.
Thus, the modelling of the relational adjective \emph{known} is quite similar to the semantics of the corresponding verb \emph{know}. It can be modelled for instance via the frame `$X$ is known to $Y$' and
reference {\tt foaf:knows} as:
 
\begin{small}\begin{verbatim}
lexicon:known a lemon:LexicalEntry ;
  lemon:canonicalForm known:Lemma ;
  lemon:sense known:Sense ;
  lemon:synBehavior known:Frame .

known:Lemma lemon:writtenRep "known"@eng .

known:Frame lexinfo:attributeArg known:Subject ;
  lexinfo:prepositionalObject known:Object .

known:Sense lemon:reference foaf:knows ;
  lemon:subjOfProp known:Subject ;
  lemon:objOfProp known:Object .
	
known:Object lemon:marker lexicon:to .
\end{verbatim}\end{small}

%\textbf{\textcolor{blue}{It is important to specify that relational adjectives can also have another definition in literature \cite{morzycki2013nonscales}. They represent a particular class of adjectives that denotes properties of kinds, rather than of simple nouns. In the case for instance of:}}
%
%\begin{enumerate}[resume,noitemsep,nolistsep]
%\item \textbf{\textcolor{blue}{Rose is a gastrointestinal surgeon}} \label{ex:gastro}
%\end{enumerate}
%
%\textbf{\textcolor{blue}{`gastrointestinal' refers to the property of a kind, not an individual. Once again, the interpretation of the adjective is context-sensitive, as clearly showed in the following examples:}}
%
%\begin{enumerate}[resume,noitemsep,nolistsep]
%\item \textbf{\textcolor{blue}{John is an industrial engineer}} \label{ex:industrial}
%\item \textbf{\textcolor{blue}{This is an industrial gas turbine}}
%\end{enumerate}



\section{Adjectives in question answering}

Of the 150 training and test questions of the QALD-4 benchmark\footnote{\url{http://www.sc.cit-ec.uni-bielefeld.de/qald/}} 
for question answering over linked data, 67 contain adjectives. 

Some of the occurring adjectives do not have a semantic contribution, or at least none that is separable from the noun, 
as exemplified in the noun phrases in~(\ref{ex:intersectives1}) and (\ref{ex:intersectives2}).\footnote{$|[\cdot|]$ stands for 
`denotes' and the prefixes \texttt{dbo} and \texttt{res} abbreviate the DBpedia namespaces \texttt{http://dbpedia.org/ontology/}
and \texttt{http://dbpedia.org/resource/}, respectively.}
\begin{enumerate}[resume,noitemsep]
\item \begin{enumerate}
 \item $|[$official website$|]=\texttt{dbo:website}$
 %\item $|[$artistic movement$|]=\texttt{dbo:movement}$ \textbf{\textcolor{blue}{subsective rather than intersective? It acquires its specific meaning in combination with the the noun it modifies, e.\,g. `an artistic movement' (meaning a group of artists) versus `an artistic mind' (a creative, art-oriented mind)}}
 \item $|[$national anthem$|]=\texttt{dbo:anthem}$
 \end{enumerate}
 \label{ex:intersectives1}
\item \begin{enumerate}
 \item $|[$official languages$|]=\texttt{dbo:officialLanguages}$
% \item $|[$American inventions$|]=\texttt{yago:AmericanIventions}$
 \item $|[$military conflicts$|]=\texttt{dbo:battle}$ 
 %\textbf{\textcolor{blue}{relational rather than intersective? (in the second proposed definition for relational adjectives? it stands for everything that goes with `military' (e.g. social disruption, economic distruction, etc.), not just weapons}}
 \end{enumerate}
 \label{ex:intersectives2}
\end{enumerate}

Otherwise, the most common kinds of adjectives among them are intersective and gradable adjectives.

All intersective adjectives denote restriction classes that are not explicitely named in DBpedia, 
in correspondence with the modelling proposed in Section~\ref{sec:intersectives} above, for example:
\begin{enumerate}[resume,noitemsep]
\item \begin{enumerate}
 \item $|[$Danish$|]=\exists\,\texttt{dbo:country}\,.\,\texttt{res:Denmark}$
 \item $|[$female$|]=\exists\,\texttt{dbo:gender}\,.\,\texttt{res:Female}$
 \item $|[$Methodist$|]=\exists\,\texttt{dbo:religion}\,.\,\texttt{res:Methodism}$
 \end{enumerate}
\end{enumerate}

In some cases these intersectives have a context-dependent and highly ontology-specific meaning, 
often tightly interwoven with the meaning of the noun, as in the following examples:
\begin{enumerate}[resume,noitemsep]
\item \begin{enumerate}
\item $|[$first president of the United States$|]=\exists\,\texttt{dbo:office}\,.\,\text{`1st President of the United States'}$
\item $|[$first season$|]=\exists\,\texttt{dbo:seasonNumber}\,.\,1$
\end{enumerate}
\end{enumerate}

All gradable adjectives that occur in the QALD-4 question set can be captured in terms of \emph{lemonOILS} 
as \texttt{CovariantScalar} (e.g. `high') or \texttt{ContravariantScalar} (e.g. `young') (cf. Section~\ref{sec:gradables} above), 
bound to a DBpedia datatype property (e.g. \texttt{elevation} or \texttt{birthDate}). 
The positive form of those adjectives only occurs in `how' questions, denoting the property they are bound to, for example:
\begin{enumerate}[resume,noitemsep] 
\item \begin{enumerate}
 \item $|[$deep$|]=\texttt{dbo:depth}$ in `How deep is Lake Placid?'
 %\item how heavy (\texttt{dbo:mass})
 \item $|[$tall$|]=\texttt{dbo:height}$ in `How tall is Michael Jordan?'
 %\item how high (\texttt{dbo:elevation})
 \end{enumerate}
\end{enumerate}
The comparative form denotes the property they are bound to, together with an aggregation operation, usually a filter 
invoking a term of comparison that depends on whether the adjective is covariant or contravariant.
\begin{enumerate}[resume]
\item \begin{enumerate}[noitemsep,nolistsep]
 \item $|[$Which mountains are higher than the Nanga Parbat?$|]=$
       \begin{small}\begin{verbatim}
       SELECT DISTINCT ?uri WHERE { 
        res:Nanga_Parbat dbo:elevation ?x .
        ?uri rdf:type dbo:Mountain .
        ?uri dbo:elevation ?y . 
        FILTER (?y > ?x) 
       }
       \end{verbatim}\end{small} 
 %\item $|[$Was the Cuban Missile Crisis earlier than the Bay of Pigs Invasion?$|]=$ 
 %      \begin{small}\begin{verbatim}
 %      ASK WHERE {
 %       res:Cuban_missile_crisis dbo:date ?x . 
 %       res:Bay_of_Pigs_Invasion dbo:date ?y . 
 %       FILTER (?x < ?y) 
 %      }
 %      \end{verbatim}\end{small}
       % duplicate
\end{enumerate}
\end{enumerate}

\vspace{-1.5em}
Finally, the superlative form denotes the property they are bound to, together with an aggregation operation, 
usually an ordering with a cut-off of all results except the first one, as exemplified in~(\ref{ex:superlative2}). 
In some cases, the superlative property is already encoded in the ontology, e.g., in the case of the property {\tt dbo:highestPlace}. 
%\footnote{The prefix \texttt{dbp} abbreviates the DBpedia namespace \texttt{http://dbpedia.org/property/}}.
\begin{enumerate}[resume,noitemsep]
\item $|[$What is the longest river?$|]=$
       \begin{small}\begin{verbatim}
       SELECT DISTINCT ?uri WHERE { 
        ?uri rdf:type dbo:River . 
        ?uri dbo:length ?l . 
       } ORDER BY DESC(?l) OFFSET 0 LIMIT 1 
       \end{verbatim}\end{small}
\label{ex:superlative2}
%\item \begin{enumerate}
%\item $|[$highest place$|]=\texttt{dbo:highestPlace}$
%\item $|[$lowest rank in the FIFA World Ranking$|]=\texttt{dbp:fifaMin}$
%\end{enumerate}
%\label{ex:superlative1}
\end{enumerate}
%In addition, there are gradable adjectives that are bound to a property that constitutes the semantic contribution 
%of the noun. Examples are `frequent' and `long'; the meaning of their superlative forms is captured in terms of 
%SPARQL aggregation operations, as shown in~\ref{ex:frequent} and~\ref{ex:span}. 
%\begin{enumerate}[resume,noitemsep,nolistsep]
%\item $|[$What is the most frequent death cause?$|]=$
%       \begin{small}\begin{verbatim}
%       SELECT DISTINCT ?uri WHERE {
%        ?x dbo:deathCause ?uri . 
%       } ORDER BY DESC( COUNT(DISTINCT ?uri) ) 
%         OFFSET 0 LIMIT 1
%       \end{verbatim}\end{small}
%       \label{ex:frequent}
%\item $|[$What is the bridge with the longest span?$|]=$
%\begin{small}\begin{verbatim}
%SELECT DISTINCT ?uri 
%WHERE {
%        ?uri rdf:type dbo:Bridge .
%        ?uri dbo:mainspan ?s . 
%} 
%ORDER BY DESC(?s) 
%OFFSET 0 LIMIT 1
%\end{verbatim}\end{small}
%\end{enumerate}
% duplicate

\vspace{-1.5em}
There are some instances of operator adjectives. Examples are `former', as in \ref{ex:former},
which does not refer to an element in the DBpedia ontology but is instead a disambiguation clue
in the given query, and `professional', which refers 
to the property \texttt{dbo:occupation}, see \ref{ex:professional}.
\begin{enumerate}[resume,noitemsep]
\item $|[$the former Dutch queen Juliana$|]=\texttt{res:Juliana}$ \label{ex:former}
\item %$|[$professional$|]=\texttt{dbo:occupation}$ (as opposed to, e.g., hobby),\\
       %for example 
    $|[$professional surfer$|]=\exists\,\texttt{dbo:occupation}\,.\,\texttt{res:Surfing}$ \label{ex:professional}
\end{enumerate}

%\textbf{\textcolor{blue}{We can argue that there are at least two interpretations for these adjectives: (a) either they are absolute gradable (on the same line of other adjective, such as `straight' or `sole'), or (b) they are a kind of privative and again operator adjectives. In fact, contrarily to their definition, they do not entail the negation of the noun property (as in the case of former president), in an explicit, but in an implicit way (`first' necessarily denies `second', and it is not gradable).}}

%\textbf{\textcolor{blue}{We miss some examples for relational adjectives. An example (as taken from DBpedia) could be: ``Give me all cars that are produced in Germany'' (`all' relates to `produced in Germany'). This the only example I found in the list of questions proposed in one of the first drafts of this paper.}}
% highest population density ?uri dbp:densityrank ?x . } ORDER BY ASC(?x) OFFSET 0 LIMIT 1
% `second' in `the second highest' (\texttt{dbo:elevation} + \texttt{ORDER BY DESC($\cdot$) OFFSET 1 LIMIT 2})

% first album (\texttt{releaseDate} + \texttt{ORDER BY ASC($\cdot$) OFFSET 0 LIMIT 1})
% past two years (\texttt{year} + \texttt{FILTER})
% alive (\texttt{deathDate} + \texttt{FILTER !BOUND})


\section{Related work}

The categorization of adjectives in terms of formal semantics goes back to Montague~\shortcite{montague1970english} and Vendler~\shortcite{vendler1968adjectives}. However, one of the most significant attempts to assign a formal meaning was carried out in the Mikrokosmos project~\cite{raskin1995lexical}. The approach to adjective modelling in the Mikrokosmos provided one of the first computational implementation of a microtheory of adjective meaning.

This was one of the first works to treat the case of a micro-theory of adjectives, in which the results were ``machine-tractable'', in that they could be formally defined by a computer. However, the applications of this were limited and no formal logic was attached to the semantic representations. Nevertheless, much of the modelling resembles ours. In particular, scalar adjectives in the Microkosmos project are modeled by association with an attribute and a range, e.g., `big' is described as being {\tt >0.75} (i.e., 75\% of all known instances) on the {\tt size-attribute}. Still, these classifications do not clearly separate meaning and syntax and also require a separate modelling of comparatives and class-specific meanings for many adjectives.

Amoia and Garden~\shortcite{amoia2006adjective} handled the problem of adjectives in the context of textual entailment. They analyzed 15 classes that show the subtle interaction between the semantic class (e.g., `privative') and the issues of attributive/predicative use and gradability.  
Abdullah and Frost~\shortcite{abdullah2005adjectives} focused on the modelling of privative adjectives by arguing that these adjectives modify the underlying set itself in a manner that is naturally second-order. Similarly, Partee~\shortcite{partee2003there} proposed a limited second-order model by means of the `head primary principle' requiring that adjectives are interpreted within their context. %, \textbf{\textcolor{blue}{here is not clear then what `head' in the principle's name stands for; is it the head noun they are bound to?}}. 
Bankston's analysis~\shortcite{bankston2003modeling}, however, shows that the fundamental nature of many adjectives is higher-order, and provides a very sophisticated formal representation framework for adjectives. A more thorough discussion of non-gradable, non-intersective adjectives is given by Morzycki~\shortcite{morzycki2013nonscales}.
Bouillion~\shortcite{bouillon1999adjective} consider the case of the French adjective `vieux' (`old'), which he interprets as selecting two different
elements in the event structure of an attributed noun, that is whether the
state, e.g., `being a mayor' for `mayor', is considered old or the individual
itself. In this way, the introduction of two senses for `vieux' is avoided, 
however it remains unclear if such reasoning introduces more complexity than
the extra senses. %\textcolor{blue}{More than Buillon's study (we could then also cite the German study of \cite{beck2000relational} or the Spanish study of \cite{mcnally2004relational}) we could spend some more words on Pustejovsky. Here what I would suggest:}}
In his analysis of adjectives, Larson \shortcite{larson1998event} suggests that many adjectives denote properties of \textit{events}, rather than of simple heads or nouns (which does not fall very far from the statement, made above, that relational adjectives denote properties of kinds). Pustejovsky \shortcite{pustejovsky1992event,pustejovsky1991generative} and Lenci \shortcite{lenci2000qualia} state that lexical and semantic decomposition can be achieved generatively, assigning to each lexical item a specific qualia structure. For instance, in an expression like:

\begin{enumerate}[resume,noitemsep,nolistsep]
\item The round, heavy, wooden, inlaid magnifying glass \label{ex:qualia}
\begin{itemize}[noitemsep]
\item `round' represents the \textit{Formal} role (giving indications of shape and dimensionality)
\item `heavy' and `wooden' is the \textit{Constitutive} role and indicates the relation between the object and its parts (e.\,g. by specifying weight, material, parts and components)
\item `inlaid' is the \textit{Agentive} role of the lexical item, denoting the factors that have been involved in the generation of the objects, such as creator, artifact, natural kind, and causal chain
\item `magnifying' describes the \textit{Telic} role of `glass', since it shows its purpose and function
\end{itemize}
\end{enumerate}

Finally, Peters and Peters~\shortcite{peters2000treatment} provide one of the few other practical reports on modelling adjectives with ontologies, in the context of the SIMPLE lexica. This work is primarily focussed on the categorization of by means of intensional and extensional properties, rather than due to their logical modelling. 

\section{Conclusion}

In this paper we have proposed an approach to model the semantics of adjectives in the context 
of the lexicon-ontology interface with a focus on the ontology-lexicon model lemon. We have argued that the semantics of adjectives, in particular gradable and privative adjectives, is beyond what can be expressed in first-order logics, OWL in particular. Instead, capturing the semantics of such adjectives requires formalisms that are non-monotonic, second-order and can represent fuzzy concepts. We have proposed an extension of lemon by the lemon OILs vocabulary that add `syntactic sugar' that allows to represent the semantics of adjectives in a way that abstracts from the actual representational formalism used. Future work will show whether this model is scalable and applicable to most adjectives as well as domains and natural languages.


%\section*{Acknowledgements}
%\vspace{-1.0em}

\bibliographystyle{acl}
\bibliography{cogalex-adjectives}


\end{document}
